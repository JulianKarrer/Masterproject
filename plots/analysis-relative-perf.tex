
\begin{figure}
\centering
\begin{tikzpicture}
    \begin{axis}[
        width=0.8\textwidth,
        height=0.5\textwidth,
        xtick      ={0.5,1,2,4,8,16}, 
        xticklabels={0.5,1,2,4,8,16}, 
        xmode = log,
        ymode = linear,
        scale only axis,
        grid=both,
        major grid style={line width=0.5pt, opacity=0.6}, 
        minor grid style={line width=0.2pt, opacity=0.3},
        legend pos=north west,
        legend cell align=left,
        xlabel={Fluid Column Height $\left[\text{m}\right]$},
        ylabel={Relative Average Performance $\sfrac{\bar{P}}{\bar{P}_0}$},
        scaled y ticks=false, 
        log ticks with fixed point,
]% dfsph-warm
\addplot[color=Spectral3, thick, opacity=1.0, mark=*] coordinates {
(0.5 , 0.8303387725527877)
(1.0 , 0.9291059566733645)
(2.0 , 0.9647962003676175)
(4.0 , 0.9016155970446461)
(8.0 , 1.1663086240153462)
(16.0 , 1.5401296899509755)
};
 \addlegendentry{DFSPH};
% iisph-cold
\addplot[color=Spectral9, thick, opacity=1.0, mark=*] coordinates {
(0.5 , 1.0)
(1.0 , 1.0)
(2.0 , 1.0)
(4.0 , 1.0)
(8.0 , 1.0)
(16.0 , 1.0)
};
 \addlegendentry{IISPH cold start};
% iisph-warm
\addplot[color=Spectral1, thick, opacity=1.0, mark=*] coordinates {
(0.5 , 1.1519971448687873)
(1.0 , 1.177017141550476)
(2.0 , 1.1750515721083972)
(4.0 , 1.0846933088459771)
(8.0 , 1.3626246492276333)
(16.0 , 1.5847764993545679)
};
 \addlegendentry{IISPH warm start};
% pcisph-3iter
\addplot[color=Spectral7, thick, opacity=1.0, mark=*] coordinates {
(0.5 , 0.6816568417319644)
(1.0 , 0.46635833652290193)
(2.0 , 0.4881766277847248)
(4.0 , 0.6136956724059859)
(8.0 , 0.6676770495955897)
(16.0 , 0.713138108226309)
};
 \addlegendentry{PCISPH};

\end{axis}
\end{tikzpicture}
\caption{The average performance score $\bar{P}$ of each solver variation as a function of the simulated fluid column's height, as shown in \autoref{fig:analysis-perf-scaling}, is rescaled by dividing each data point by the result for the cold-started IISPH solver $\hat{P}_0$, the behaviour of which defines the value $1$ in this graph. As a result, the relative performances as percentages of this baseline IISPH implementation can be observed for varying scene complexity. It again becomes clear that for the tested scenes, the equation-of-state-based PCISPH consistently performs worse than its counterparts that explicitly solve the pressure Poisson equation on a per-particle basis. While the divergence-free source term of DFSPH seems to result in more overhead than it is worth for low complexities, it results in a performance increase for more demanding scenes, as is intended, eventually catching up to the warm-started IISPH solver. IISPH solver performance again consistently improves across all considered scenes when a warm start is employed, resulting in the highest performances measured each time.}
\label{fig:analysis-relative-perf}
\end{figure}
